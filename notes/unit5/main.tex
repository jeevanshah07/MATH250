\section{Unit 5}
\subsection{Lecture 19: Eigenvalues and Eigenvectors}
\begin{defbox}{}{}{}
    Let $A$ be any $n\times\,n$ matrix. A \underline{non-zero} vector $\vec{S}$ is called an \textbf{eigenvector} of $A$ is $A\vec{S} = \lambda\vec{S}$ for $\lambda \in \CC$. Such $\lambda$ is called the corresponding \textbf{eigenvalue} of $A$.
\end{defbox}

\begin{example}{}{}
    Given $A$ and $\vec{S}$ below, verify $\vec{S}$ is an eigenvector of $A$ and find the corresponding eigenvalue of $A$.  
    \[
        A = \begin{pmatrix}
            -5 & -4 \\ 8 & 7
        \end{pmatrix}
        \text{ and } 
        \vec{S} = \begin{pmatrix}
            1 \\ -2
        \end{pmatrix}
    \]
    \begin{solution}
        \begin{align*}
            AS &= \begin{pmatrix}
                -5 & -4 \\ 8 & 7
            \end{pmatrix}
            \begin{pmatrix}
                1 \\ -2
            \end{pmatrix} \\
            &= \begin{pmatrix}
                3 \\ -6
            \end{pmatrix} \\
            &= 3\vec{S} \\
            &\Rightarrow \boxed{\lambda = 3}
        \end{align*}
    \end{solution}
\end{example}

\begin{example}{}{}
    Given $A$ below and $\lambda=2$, find the eigenvector
    \[
        A = \begin{pmatrix}
            5 & -1 \\ 3 & 1
        \end{pmatrix} 
    \]
    \begin{solution}
        Since $\lambda$ is an eigenvalue it must satisfy the equation 
        \[
            A\vec{S} = 2\vec{S} \Rightarrow A\vec{S} - 2\vec{S} = 0
        \]
        However, since $2$ is a scalar we can't factor out $\vec{S}$. We can fix this by multiplying by the identity matrix $I$. This gives
        \[
            I\left(A\vec{S} - 2\vec{S}\right) = AI\vec{S} - 2I\vec{S} = \left(A - 2I\right)\vec{S} = 0.
        \]
        Since $\vec{S}$ is non-zero by definition, we know that $A-2I=0$. So, 
        \begin{align*}
            \begin{pmatrix}
                5 & -1 \\ 3 & 1
            \end{pmatrix} - 2\begin{pmatrix}
                1 & 0 \\ 0 & 1
            \end{pmatrix}
            &= \begin{pmatrix}
                3 & -1 \\ 3 & -1
            \end{pmatrix}
            \xrightarrow{R_2 \to R_2-R_1}
            \begin{pmatrix}
                3 & -1 \\ 0 & 0
            \end{pmatrix} \\
            &\Rightarrow X = t\begin{pmatrix}
                1 \\ 3
            \end{pmatrix}.
        \end{align*}
        Thus, our eigenvector is 
        \[
            \boxed{\begin{pmatrix}
                1 \\ 3
            \end{pmatrix}} 
        \]
    \end{solution}
\end{example}

You may notice that in the above example the solution to $A-2I$ was 
\[
    t\begin{pmatrix}
        1 \\ 3
    \end{pmatrix}.
\]
Since this generates vectors for all non-zero values of $t$ we call this the \textbf{eigenspace}.


Its important to realize that for any $n\times\,n$ matrix $A$, if $\det A=0$ then $\det A-\lambda\,I = 0$. If $\det A = 0$ then $\rank A < n$ implies that the equation $A\mathbf{x} = 0$ has \textit{non-trivial} solutions which is exactly what we are in search of. From this discussion we can now learn about the characteristic equation. 

\begin{defbox}{The Characteristic Equation}{}{}
    The \textbf{characteristic equation} of an $n\times\,n$ matrix $A$ is 
    \[
        \det A - \lambda\,I=0 
    \]
    and its solutions are the \underline{eigenvalues} of $A$.
\end{defbox}

It follows from the above definition that any characteristic equation will be a polynomial of degree $n$ and thus $A$ will have $n$ eigenvectors (including multiplicity). \\ To see this in example, lets determine the characteristic equation for any $2\times\,2$ matrix. Let $A$ be the $2\times\,2$ matrix below
\[
    A = \begin{pmatrix}
        a & b \\ c & d
    \end{pmatrix}.
\]
We can find the characteristic equation using the formula provided in the definition to give, 
\[
    \det A - \lambda\,I_2 = \begin{vmatrix}
        a - \lambda & b \\
        c & d - \lambda
    \end{vmatrix} 
    = \left(a-\lambda\right)\left(d-\lambda\right) - bc = \lambda^2 - \lambda(a+d) + (ad - bc).
\]
This process can be repeated for any $n\times\,n$ matrix to find its general characteristic equation.

\begin{impbox}{How to find Eigenvectors and Eigenvalues}{}
    Let $A$ be an $n\times\,n$ matrix. To find the eigenvalues and eigenvectors of $A$: 
    \begin{enumerate}
        \item Find the eigenvalues by solving $\det\left(A-\lambda\,I\right)=0$ (the characteristic equation)
        \item For each eigenvalue, find the corresponding eigenvector(s) y solving $A - \lambda\,I = 0$.
    \end{enumerate}
\end{impbox}

\begin{example}{}{}
    Find the eigenvalues and eigenvectors of $A$ below 
    \[
        A = \begin{pmatrix}
            5 & -1 \\ 3 & 1
        \end{pmatrix} 
    \]
    \begin{solution}
        We can find the characteristic equation by solving $\det A - \lambda\,I = 0$. This gives 
        \[
            \begin{vmatrix}
                5-\lambda & -1 \\ 3 & 1-\lambda
            \end{vmatrix} 
            = \lambda^2 - 6\lambda + 8 = 0 \Rightarrow (\lambda-4)(\lambda-2) = 0 \Rightarrow \lambda_1 = 4, \lambda_2 = 2.
        \]
        Thus we can find our eigenvectors:
        \begin{align*}
            &\Rightarrow A - 4I = \begin{pmatrix}
                1 & -1 \\ 3 & -3
            \end{pmatrix} \xrightarrow{R_2 \to R_2 - 3R_1} \begin{pmatrix}
                1 & -1 \\ 0 & 0
            \end{pmatrix}
            \Rightarrow X = t\begin{pmatrix}
                1 \\ 1
            \end{pmatrix} \\
            &\Rightarrow A - 2I = \begin{pmatrix}
                3 & -1 \\ 3 & -1 
            \end{pmatrix}
            \xrightarrow{R_2 \to R_2 - R_1} \begin{pmatrix}
                3 & -1 \\ 0 & 0
            \end{pmatrix}
            \Rightarrow X = t\begin{pmatrix}
                1 \\ 3
            \end{pmatrix}
        \end{align*}
        Therefore for all $t \neq 0$ our eigenvectors are 
        \[
            t\begin{pmatrix}
                1 \\ 1
            \end{pmatrix} \text{ and } t\begin{pmatrix}
                1 \\ 3
            \end{pmatrix}
        \]
    \end{solution}
\end{example}

\begin{defbox}{Eigenspace}{}
    The \textbf{eigenspace} is the null space of $A-\lambda\,I$ \underline{without} the zero vectors
\end{defbox}

\begin{example}{}{}
    If $S_1 = \left(\begin{smallmatrix}
        1 \\ 3
    \end{smallmatrix}\right)$ for $\lambda_1 = 2$ and $S_2 = \left(\begin{smallmatrix}
        1 \\ 1
    \end{smallmatrix}\right)$ for $\lambda_2 = 4$ what are the eigenspaces?
    \begin{solution}
        The eigenspace associated with $\lambda_1$ is 
        \[
            \Span{\begin{pmatrix}
                1 \\ 3
            \end{pmatrix}} \setminus \braces{\vec{0}}.
        \]
        The eigenspace associated with $\lambda_2$ is 
        \[
            \Span{\begin{pmatrix}
                1 \\ 1
            \end{pmatrix}} \setminus \braces{\vec{0}}.
        \]
    \end{solution}
\end{example}

Note that the symbol $\setminus$ is the set minus symbol and all it means is exclusion from the set. So $\Span{x} \setminus \braces{\vec{0}}$ would be read as `the span of the vector $\vec{x}$ withouth the zero vector'.