\section{Unit 4}
\subsection{Lecture 14: Vector Spaces and Subspaces [2.8/4.1]}

Consider $\RR^{n}$ and vectors $\vec{u}, \vec{v} \in \RR^{n}$ and notice that the following properties always hold true:
\begin{enumerate}
    \item $\forall\,\vec{v}, \vec{u} \in \RR^{n}, \vec{u} + \vec{v} \in \RR^{n}$ (Closure under addition)
    \item $\forall\,\vec{v} \in \RR^{n}\,,\forall\,\lambda \in \RR, \lambda\vec{v} \in \RR^{n}$ (Closure under scalar multiplication)
    \item $\vec{0} \in \RR^{n}$ (Inclusion of the zero vector)
\end{enumerate}
If you are unaware, the symbol $\forall$ means `for all' and the symbold $\in$ means `contained in'. So, property (1) would be read `for all vectors $v$ and $u$ in $\RR^{n}$, their sum is also in $\RR^{n}$'. \\ Now these properties that $\RR^{n}$ has make it special. In fact because of these three properties we call $\RR^{n}$ a \textbf{vector space}. 

\begin{defbox}{Subspaces of $\RR^{n}$}{}
    Consider a subset of $\RR^{n}$, $W$. $W$ is a subspace of $\RR^{n}$ if 
    \begin{enumerate}
        \item $\forall\,\vec{v}, \vec{u} \in W, \vec{u} + \vec{v} \in W$ (Closure under addition)
        \item $\forall\,\vec{v} \in W\,,\forall\,\lambda \in \RR, \lambda\vec{v} \in W$ (Closure under scalar multiplication)
        \item $\vec{0} \in W$ (Inclusion of the zero vector)
    \end{enumerate}
\end{defbox}

It's important to note that the \textit{trivial} subspaces of $\RR^n$ are $\RR^n$ itself and the zero subspace: $\braces{\vec{0}}$

\begin{example}{}{}
    If $V \subseteq \RR^{2} = \braces{\vec{v} \in \RR^{2} \mid \vec{v} \text{ is a unit vector}}$ (read `$V$ is a subset of $\RR^{2}$ such that $V$ is the set of all vectors in $\RR^{2}$ that are unit vectors'), is $V$ a subspace of $\RR^{2}$?

    \begin{solution}
        If we were to write out $V$ we would have 
        \[
            V = \braces{\begin{pmatrix}
                1 \\ 0
            \end{pmatrix}, \begin{pmatrix}
                0 \\ 1
            \end{pmatrix}} 
        \]        
        as these are the only unit vectors in $\RR^2$. From this it should be very clear that $\vec{0} \not\in V \therefore V$ is \textbf{not} a subspace of $\RR^2$.
    \end{solution}
\end{example}

\begin{example}{}{}
    If $V \subseteq \RR^{2} = \braces{\begin{pmatrix} x_1 \\ x_2 \end{pmatrix} \in \RR^{2} \mid x_1 + x_2 = 0}$, is $V$ a subspace of $\RR^2$?

    \begin{solution}
        Consider the parameterization $x_2 = t$ and $x_1 = 1-t$. Then 
        \[
            X = \begin{pmatrix}
                1 - t \\ t
            \end{pmatrix} 
            = 
            \begin{pmatrix}
                1 \\ 0
            \end{pmatrix}
            + 
            t\begin{pmatrix}
                -1 \\ 1
            \end{pmatrix}
        \]
        Once again, from this parameterization it should be easy to see that $\vec{0} \not\in V \therefore V$ is \textbf{not} a subspace of $\RR^2$.
    \end{solution}
\end{example}

\begin{example}{}{}
    Let $V = \braces{\begin{pmatrix} x_1 \\ x_2 \\ x_3 \end{pmatrix} \in \RR^{3} \mid x_1 + 2x_2 - x_3 = 0}$. Is $V$ a subspace of $\RR^{3}$?

    \begin{solution}
        Just like before consider the parameterization $x_2 = s, x_3 = t, x_1 = -2s + t$. This gives 
        \[
            X = \begin{pmatrix}
                -2s + t \\ s \\ t
            \end{pmatrix} 
            = 
            s\begin{pmatrix}
                -2 \\ 1 \\ 0
            \end{pmatrix}
            + 
            t\begin{pmatrix}
                1 \\ 0 \\ 1
            \end{pmatrix}
            = 
            \Span{\begin{pmatrix}
                -2 \\ 1 \\ 0
            \end{pmatrix}, 
            \begin{pmatrix}
                1 \\ 0 \\1
            \end{pmatrix}}
        \]
        From this we can see that all three properties are satisfied thus $V$ \textbf{is} a subspace of $\RR^{2}$. 
    \end{solution}
\end{example}

You may notice that the above example invovles an equality to the Span of a set of vectors, despite said span not being mentioned once. Well, it is included because of the following theorem:

\begin{thm}{}{}
    The span of a set of vectors from $\RR^{n}$ is a subspace of $\RR^{n}$. 
    \begin{proof}
        Consider two vectors $\vec{v}, \vec{u} \in \RR^{n}$ and the span, $\Span{x_1, x_2, \cdots\,, x_k}$. Then $\vec{v}$ and $\vec{u}$ can be expressed as a linear combination as follows:
        \begin{align*}
            \vec{u} &= c_1\vec{x_1} + c_2\vec{x_2} + \cdots + c_k\vec{x_k} \\
            \vec{v} &= \tilde{c_1}\vec{x_1} + \tilde{c_2}\vec{x_2} + \cdots + \tilde{c_k}\vec{x_k} \\
            \Rightarrow \vec{u} + \vec{v} &= \left(c_1 + \tilde{c_1}\right)\vec{x_1} + \left(c_2 + \tilde{c_2}\right)\vec{x_2} + \cdots + \left(c_k + \tilde{c_k}\right)\vec{x_k}
        \end{align*}
        Then, $\vec{u}, \vec{v}, \vec{u+v} \in \Span{x_1, x_2, \cdots\, x_k}$ by the definition of span.
    \end{proof} 
\end{thm}

\begin{example}{}{}
    Consider the set of all polynomials of degree $3, P_3$. Is $P_3$ a vector space?

    \begin{solution}
        Let $P$ be any polynomial of degree $3$. Then 
        \[
            P\left(x\right) = a_0 + a_1x + a_2x^2 + a_3x^3 
        \]
        We can have $P\left(x\right) = 0$ if $\forall x\, a_0=a_1=a_2=a_3=0$, therefore $\vec{0} \in P_3$. \\ Now consider another polynomial $Q$ of degree three:
        \[
            Q\left(x\right) = b_0 + b_1x + b_2x^2 + b_3x^3 
        \]
        Then, 
        \[
            R\left(x\right) = P\left(x\right) + Q\left(x\right) = \left(a_0 + b_0\right) + \left(a_1 + b_1\right)x + \left(a_2 + b_2\right)x^2 + \left(a_3 + b_3\right)x^3 
        \]
        But, $R$ is still a degree $3$ polynomial, so $R\in\,P_3$, and thus $P_3$ is closed under addition. We can also consider any scalar $\lambda \in \RR$. Then 
        \[
            Q\left(x\right) = \lambda\,P\left(x\right) = \lambda\,a_0 + \lambda\,a_1x + \lambda\,a_2x^2 + \lambda\,a_3x^3 
        \]
        Again, $Q \in P_3$ because $Q$ is still a degree $3$ polynomial. Thus $P_3$ \textbf{is} a vector space. 
    \end{solution}
\end{example}

One interesting thing comes when we consider vectors in $\RR^4$ and $P_3$. Notice that the general form for a vector in $\RR^{4}$ is 
\[
    \begin{pmatrix}
        x_1 \\ x_2 \\ x_3 \\ x_4
    \end{pmatrix}
    = 
    x_1 \begin{pmatrix}
        1 \\ 0 \\ 0 \\ 0
    \end{pmatrix}
    +
    x_2 \begin{pmatrix}
        0 \\ 1 \\ 0 \\ 0
    \end{pmatrix}
    + 
    x_3 \begin{pmatrix}
        0 \\ 0 \\ 1 \\ 0
    \end{pmatrix}
    +
    x_4 \begin{pmatrix}
        0 \\ 0 \\ 0 \\ 1
    \end{pmatrix}
\]
Now, consider the general (vector) form of any polynomial in $P_3$: 
\[
    \begin{pmatrix}
        a_0 \\ a_1x \\ a_2x^2 \\ a_3x^3
    \end{pmatrix}
    = 
    a_0 \begin{pmatrix}
        1 \\ 0 \\ 0 \\ 0
    \end{pmatrix}
    + 
    a_1 \begin{pmatrix}
        0 \\ x \\ 0 \\ 0
    \end{pmatrix}
    + 
    a_2 \begin{pmatrix}
        0 \\ 0 \\ x^2 \\ 0
    \end{pmatrix}
    + 
    a_3 \begin{pmatrix}
        0 \\ 0 \\ 0 \\ x^3
    \end{pmatrix}
\]
Now, hopefully, you notice the extremely similarities in the structures of these two vector spaces. In fact, the structure of each of these vector spaces is considered identical in mathematics and $\RR^4$ is call \textbf{isomorphic} to $P_3$. We can show this by defining a bijective map $\phi: \RR^{4} \to P_3$. Obviously, in this case, $\phi$ is simply the direct mapping of each corresponding vector. \\ Now, if we consider these vectors \textit{without} their coefficients, what we get is actually called the \textbf{standard basis vectors} since 
\begin{align*}
    \RR^4 &= \Span{\begin{pmatrix}
        1 \\ 0 \\ 0 \\ 0
    \end{pmatrix}, 
    \begin{pmatrix}
        0 \\ 1 \\ 0 \\ 0
    \end{pmatrix}, 
    \begin{pmatrix}
        0 \\ 0 \\ 1 \\ 0
    \end{pmatrix}, 
    \begin{pmatrix}
        0 \\ 0 \\ 0 \\ 1
    \end{pmatrix}} \\
    P_3 &= \Span{\begin{pmatrix}
        1 \\ 0 \\ 0 \\ 0
    \end{pmatrix}, 
    \begin{pmatrix}
        0 \\ x \\ 0 \\ 0
    \end{pmatrix}, 
    \begin{pmatrix}
        0 \\ 0 \\ x^2 \\ 0
    \end{pmatrix}, 
    \begin{pmatrix}
        0 \\ 0 \\ 0 \\ x^3
    \end{pmatrix}}
\end{align*}

\begin{example}
    Consider $W$ such that $W = \Span{\vec{u}, \vec{v}}$. Find $\vec{u}$ and $\vec{v}$.
    \[
        W = \begin{pmatrix}
            5b + 2c \\ b \\ b
        \end{pmatrix}  = 
        \Span{\vec{u}, \vec{v}}
    \]

    \begin{solution}
        \[
            W = b\begin{pmatrix}
                5 \\ 1 \\ 0
            \end{pmatrix} 
            +
            c\begin{pmatrix}
                2 \\ 0 \\1
            \end{pmatrix}
            =
            \Span{\begin{pmatrix}
                5 \\ 1 \\ 0 
            \end{pmatrix}, 
            \begin{pmatrix}
                2 \\ 0 \\ 1
            \end{pmatrix}}
            \Rightarrow
            \vec{u} = \begin{pmatrix}
                5 \\ 1 \\ 0
            \end{pmatrix}
            \text{ and }
            \vec{v} = \begin{pmatrix}
                2 \\ 0 \\ 1
            \end{pmatrix}
        \]
    \end{solution}
\end{example}

We can as well notice that the above span was a subspace of $\RR^3$ since it's span was of a set of vectors contained in $\RR^{3}$.

\begin{example}{}{}
    Consider $v_1, v_2, v_3$ and $W$ as below. 
    \[
        v_1 = \begin{pmatrix}
            1 \\ 0 \\ -1
        \end{pmatrix} 
        \quad 
        v_2 = \begin{pmatrix}
            2 \\ 1 \\ 3
        \end{pmatrix}
        \quad
        v_3 = \begin{pmatrix}
            4 \\ 2 \\ 6
        \end{pmatrix}
        \quad 
        W = \begin{pmatrix}
            3 \\ 1 \\ 2
        \end{pmatrix}
    \] 
    \begin{parts}
        \part Is $W \in \Span{v_1, v_2, v_3}$?
        \begin{solution}
            Consider $(A \mid W)$ where $A = \begin{pmatrix}
                \left(\vec{v_1}\right) & \left(\vec{v_2}\right) & \left(\vec{v_3}\right)
            \end{pmatrix}$
            \[
                \begin{Lamatrix}{4}{3}
                    1 & 2 & 4 & 3 \\
                    0 & 1 & 2 & 1 \\
                    -1 & 3 & 6 & 2 
                \end{Lamatrix} 
                \xrightarrow{R_3 \to R_3 + R_1}
                \begin{Lamatrix}{4}{3}
                    1 & 2 & 4 & 3 \\
                    0 & 1 & 2 & 1 \\
                    0 & 5 & 10 & 5 
                \end{Lamatrix} 
                \xrightarrow{R_3 \to R_3 + 5R_2}
                \begin{Lamatrix}{4}{3}
                    1 & 2 & 4 & 3 \\
                    0 & 1 & 2 & 1 \\
                    0 & 0 & 0 & 0 
                \end{Lamatrix} 
            \]
            Thus, the system is consistent so $W \in \Span{v_1, v_2, v_3}$
        \end{solution}

        \part Is $\braces{v_1, v_2, v_3}$ linearly independent?
        \begin{solution}
            From the above part we know that $v_3 = 2v_2 \therefore$ the set is linearly \textbf{depdendent}. 
        \end{solution}
    \end{parts}
\end{example}

\begin{defbox}{Basis of a Subspace}{}
    The basis of a subspace $W$ of $\RR^{n}$ is any set of linearly independent vectors such that its span is $W$.
\end{defbox}