\section{Unit 1}
\subsection{Systems of Linear Equations}

A system of linear equations is when you have more than 1 equation with more than 1 unknown.

\begin{example}{}{}
    Solve the following system:
    \[
        \begin{cases}
            x_1 + 2x_2 &= 5 \\
            2x_1 + x_2 &= 4
        \end{cases}
    \]
    \begin{align*}
        x_1 + 2x_2 = 5 &\Rightarrow x_1 = 5-2x_2 \\
        2x_1 + x_2 = 4 &\Rightarrow 2(5-2x_2) + x_2 = 4 \Rightarrow 10-3x_2=4 \\ 
        &\Rightarrow x_2=2 \\
        &\Rightarrow x_1 = 1
    \end{align*}
\end{example}

Note that the solution to the previous system $(x_1, x_2) = (1, 2)$ also corresponds to the point of intersection of the lines that each equation represents. This then implies that non-parallel lines have a single solution, paralell lines have no solutions, and scalar multiples of the same line have infinitely many solutions.

\begin{example}{Systems with Different Number of Solutions}{}
    \begin{enumerate}
        \item One Solution:
        \[
            \begin{cases}
                x_1 + 2x_2 &= 5 \\
                2x_1 + x_2 &= 4
            \end{cases} 
        \]

        \item No solution:
        \[
            \begin{cases}
                x_1 + 2x_2 &= 5 \\
                2x_1 + 4x_2 &= 6
            \end{cases}
        \]

        \item Infinite Solutions
        \[
            \begin{cases}
                x_1 + 2x_2 &= 5 \\
                2x_1 + 4x_2 &= 10
            \end{cases}
        \]
    \end{enumerate}
\end{example}

\begin{impbox}{Elementary Row Operations}
    Consider the system 
    \[\begin{cases}
        &E_1 \\
        &E_2 \\
        &\vdots \\
        &E_n
    \end{cases}\]
    Where $E_i$ is the $i$-th equation. Then the following operations on the system \textbf{do not change the solution}
    \begin{enumerate}
        \item Changing the order of equations: $E_i \leftrightarrow E_j$ where $i \ne j$ for the $i$-th and $j$-th equation
        \item Scaling equations: $E_i \rightarrow \lambda E_i$ where $\lambda\in\RR\left(\lambda \ne 0\right)$
        \item Combining equations: $E_i \rightarrow E_i + \lambda E_j$ where $i \ne j$ and $\lambda \in \RR \left(\lambda \ne 0\right)$
    \end{enumerate}
\end{impbox}

\begin{example}{}{}
    Solve the following system:
    \[\begin{cases}
        E_1 &= x_1 + 2x_2 = 5 \\
        E_2 &= 2x_1 + x_2 = 4
    \end{cases}\]
    \begin{align*}
        -2E_1 &= -2x_1 - 4x_2 = -10 \\
        E_2 - 2E_1 &= (2x_1 - x_2) - 2x_1 - 4x_2 = 4-10 \\
        &\Rightarrow -3x_2 = -6 \\
        &\Rightarrow x_2 = 2 \\
        &\Rightarrow x_1 = 1
    \end{align*}
\end{example}

\begin{defbox}{Matricies}
    A \textbf{matrix} is a rectangular array of numbers arranged in rows and columns. A matrix $A$ has size $m \times n$ if it has $m$ \textbf{rows} and $n$ \textbf{columns}. A matrix is written as
    \[\underset{m \times n}{A} = \begin{pmatrix}
    a_{11} & a_{12} & \dots & a_{1n} \\
    a_{21} & a_{22} & \dots & a_{2n} \\
    \vdots & & \ddots & \vdots \\
    a_{m1} & a_{m2} & \dots & a_{mn}
    \end{pmatrix}\]
    Where each $a_{ij}$ is called an element of the matrix. $i$ denotes the row and $n$ denotes the column of the element.
\end{defbox}

With this in mind, note that we can represent the previous system as a matrix:
\[\begin{pmatrix}
    1 & 2 & 5 \\
    2 & 1 & 4
\end{pmatrix}\]
where each row represents an equation and each element corresponds to either the coefficent of a variable or the solution. With this system in matrix form we can manipulate the rows as follows:
\[
    \begin{pmatrix}
        1 & 2 & 5 \\
        2 & 1 & 4
    \end{pmatrix} 
    \xrightarrow[R_2 \rightarrow R_2 - 2R_1]{} 
    \begin{pmatrix}
        1 & 2 & 5 \\
        0 & -3 & -6
    \end{pmatrix} 
\]
Since each row corresponds to an equation, if we take a look at the bottom row we can see 
\[ -3x_2 = -6 \Rightarrow x_2 = 2 \Rightarrow x_1 = 1\]
Thus allowing us to reach the same final answer with a lot less hassle. 

\begin{impbox}{General Form of a System of Linear Equations}
    Consider any system of linear equations in the form 
    \[
        \begin{cases}
            a_{11}x_{1} + a_{12}x_{2} + \dots + a_{1n}x_n &= b_1 \\
            a_{21}x_{1} + a_{22}x_{2} + \dots + a_{2n}x_n &= b_2 \\
            \vdots \\
            a_{m1}x_{1} + a_{m2}x_{2} + \dots + a_{mn}x_n &= b_m \\
        \end{cases} 
    \]
    Now, notice we can rewrite this as a matrix:
    \[
        A = \begin{amatrix}{4}
            a_{11} & a_{12} & \dots & a_{1n} & b_1 \\
            a_{21} & a_{22} & \dots & a_{2n} & b_2 \\
            \vdots & & \ddots & \vdots & \vdots \\
            a_{m1} & a_{m2} & \dots & a_{mn} & b_m
        \end{amatrix}
    \]
    Everything to the \textbf{left} of the solid line is refered to as the \textit{coefficent matrix}. $A$ itself is referred to as the \textbf{augmented matrix} for the system.
\end{impbox}

\begin{defbox}{Row Echelon Form}
    Given a matrix $A$, $A_{\text{REF}}$ is an equivalent matrix that satisfies the following properites:
    \begin{enumerate}
        \item All zero rows are below non-zero rows
        \item each next leading element is in the column to the right of the previous leading element (called pivots)
    \end{enumerate}
    Note that the \textbf{leading element} of a row is simply the first non-zero element in that row. 
\end{defbox}

A matrix can also be put in RREF (reduced row echelon form) if it is alreay in REF, each pivot is $1$, and the only non-zero element in the pivot column is the pivot. This would be $A_{\text{RREF}}$. Now, lets try to combine all we've done by applying basic ERO to a simple matrix to put it in REF. 


\begin{example}{Basic REF}{ref}
    Consider the system 
    \[
        \begin{cases}
            x_1 + 2x_2 &= 5 \\
            2x_1 + x_2 = 4
        \end{cases} 
    \]
    It's augmented matrix can be put into REF as follows
    \[
        \begin{amatrix}{1}
            A & B
        \end{amatrix}
        = 
        \begin{amatrix}{2}
            1 & 2 & 5 \\
            2 & 1 & 4
        \end{amatrix}
        \xrightarrow[R_2 \rightarrow R_2 - 2R_1]{} 
        \begin{amatrix}{2}
            1 & 2 & 5 \\
            0 & -3 & -6
        \end{amatrix}
        = 
        \begin{amatrix}{1}
            A & B
        \end{amatrix}_{\text{REF}}
    \]
\end{example}

\begin{example}{Basic RREF}{}
    Consider again the system from ex~(\ref{th:ref}) and notice we can put it into RREF as follows:
    \[
        \begin{amatrix}{1}
            A & B
        \end{amatrix}_{\text{REF}}
        =  
        \begin{amatrix}{2}
            1 & 2 & 5 \\
            0 & -3 & -6
        \end{amatrix}
        \xrightarrow[R_2 \rightarrow -\frac{1}{3}R_2]{} 
        \begin{amatrix}{2}
            1 & 2 & 5 \\
            0 & 1 & 2
        \end{amatrix}
        \xrightarrow[R_1 \rightarrow R_1 - 2R_2]{} 
        \begin{amatrix}{2}
            1 & 0 & 1 \\
            0 & 1 & 2
        \end{amatrix}
        = 
        \begin{amatrix}{1}
            A & B
        \end{amatrix}_{\text{RREF}}
    \]
\end{example}

It's important to note that the RREF for any given matrix is \textit{unique}.

\begin{example}{}{}
    \begin{align*}
        \begin{cases}
            x_1 + 2x_2 - x_3 &= 2 \\
            2x_1 + x_2 + x_3 &= 1 \\ 
            x_1 - x_2 + 2x_3 &= -1
        \end{cases}
        = 
        \begin{amatrix}{3}
            1 & 2 & -1 & 2 \\
            2 & 1 & 1 & 1 \\
            1 & -1 & 2 & -1
        \end{amatrix} &\xrightarrow[\substack{R_2 \to R_2 - 2R_1 \\ R_3 \to R_3 - R_1}]{} 
        \begin{amatrix}{3}
            1 & 2 & -1 & 2 \\
            0 & -3 & 3 & -3 \\
            0 & -3 & 3 & -3
        \end{amatrix} \\
        &\xrightarrow[R_3 \to R_3 - R_2]{}
        \begin{amatrix}{3}
            1 & 2 & -1 & 2 \\
            0 & -3 & 3 & -3 \\
            0 & 0 & 0 & 0
        \end{amatrix}
    \end{align*}
\end{example}