\documentclass{exam}
\usepackage{../../shah250}

\hypersetup{colorlinks=true, linktoc=section, linkcolor=blue}

\pagestyle{headandfoot}
\firstpageheadrule
\runningheadrule
\firstpageheader{Prof. Shtelen \\ Linear Algebra}{Homework 7}{Jeevan Shah}
\runningheader{Linear Algebra \\ Homework 7}{}{Shah}
\firstpagefooter{}{}{}
\runningfooter{ }{\thepage}{ }

\printanswers

\begin{document}
\colorbox{red}{\underline{\#3 [6.5]:}} Let $A$ and $\vec{b}$ be the matricies below for the system $A\vec{x} = \vec{b}$
\[
    A = \begin{pmatrix}
        1 & -2 \\
        -1 & 2 \\
        0 & 3 \\
        2 & 5
    \end{pmatrix},
    \quad \vec{b} = \begin{pmatrix}
        3 \\ 1 \\ -4 \\ 2
    \end{pmatrix}
\]
\begin{parts}
    \part Construct the normal equation for $\hat{x}$
    \begin{solution}
        Begin by noticing that the system is inconsistent and so we must approximate the solution using the least squares method: $A^{T}A\hat{x} = A^{T}\vec{b}$ where $\hat{x}$ is the vector such that $\|b-A\hat{x}\| \leq \|b - A\vec{x}\|$ for all other $\vec{x}$. We start by finding $A^{T}A$ and $A^{T}\vec{b}$: 
        \begin{align*}
            A^{T}A &= \begin{pmatrix}
                6 & 6 \\
                6 & 42
            \end{pmatrix} \\
            A^{T}\vec{b} &= \begin{pmatrix}
                6 \\ -6
            \end{pmatrix}
        \end{align*}
        We can now setup the normal equations for $\hat{x}$:
        \[
            \boxed{
                \begin{pmatrix}
                    6 & 6 \\
                    6 & 42
                \end{pmatrix}
                \begin{pmatrix}
                    x_1 \\ x_2
                \end{pmatrix}
                = \begin{pmatrix}
                    6 \\ -6
                \end{pmatrix}
            } 
        \]
    \end{solution}

    \part Solve for $\hat{x}$
    \begin{solution}
        To solve for $\hat{x}$ we can first notice that $\left(A^{T}A\right)^{-1}$ exists since $\det{A^{T}A} = 216 \neq 0$. Thus, using the formula for the inverse of a $2\times\,2$ matrix we have 
        \[
            \left(A^{T}A\right)^{-1} = \frac{1}{216}\begin{pmatrix}
                42 & -6 \\
                -6 & 6
            \end{pmatrix} .
        \]
        Now, 
        \[
            A^{T}A\hat{x} = A^{T}\vec{b} \Rightarrow \hat{x} = \left(A^{T}A\right)^{-1}\left(A^T\vec{b}\right).
        \]
        Thus, using our calculation of $A^{T}\vec{b}$ from part (a), we have 
        \[
            \hat{x} = \frac{1}{216}\begin{pmatrix}
                42 & -6 \\
                -6 & 6
            \end{pmatrix}
            \begin{pmatrix}
                6 \\ -6 
            \end{pmatrix}
            = \frac{1}{216}\begin{pmatrix}
                288 \\ -72
            \end{pmatrix}
            = \boxed{
                \begin{pmatrix}
                    4/3 \\ -1/3
                \end{pmatrix}
            }
        \]
    \end{solution}

\end{parts}

\colorbox{red}{\underline{\#7 [7.1]:}} Determine if the matrix below is orthogonal. Is it is, find the inverse.
\[
    \begin{pmatrix}
        .6 & .8 \\
        .8 & -.6 
    \end{pmatrix}
\]
\begin{solution}
    Recall the definition of an orthogonal matrix is a matrix $Q$ such that $Q^{T}Q = I$. We can check if the above matrix is orthogonal using this definition: 
    \[
        \begin{pmatrix}
            .6 & .8 \\
            .8 & -.6 
        \end{pmatrix}
        \begin{pmatrix}
            .6 & .8 \\
            .8 & -.6 
        \end{pmatrix}
        = \begin{pmatrix}
            1 & 0 \\
            0 & 1
        \end{pmatrix}.
    \]
    Therefore, by definition, the above matrix \textbf{is orthogonal}. Thus, the inverse is simply the transpose, if the above matrix is $Q$, then 
    \[
        Q^{T} = \boxed{
            Q^{-1} =
            \begin{pmatrix}
                .6 & .8 \\
                .8 & -.6 
            \end{pmatrix}
        }
    \]
\end{solution}

\colorbox{red}{\underline{\#17 [7.1]:}} Orthogonally diagonalize the matrix given below with eigenvalues $\lambda = -4, 4, 7$
\[
    A = \begin{pmatrix}
        1 & 1 & 5 \\
        1 & 5 & 1 \\
        5 & 1 & 1
    \end{pmatrix}
\]
\begin{solution}
    We start by finding the eigenvectors. For $\lambda = -4$ we have 
    \begin{align*}
        A + 4I = \begin{pmatrix}
            5 & 1 & 5 \\
            1 & 9 & 1 \\
            5 & 1 & 4
        \end{pmatrix} &\xrightarrow{R_1 \leftrightarrow R_2}
        \begin{pmatrix}
            1 & 9 & 1 \\
            5 & 1 & 5 \\
            5 & 1 & 5
        \end{pmatrix}
        \xrightarrow{R_3 \to R_3 - R_2}
        \begin{pmatrix}
            1 & 9 & 1 \\
            5 & 1 & 5 \\
            0 & 0 & 0
        \end{pmatrix} 
        \xrightarrow{R_2 \to R_2 - 5R_1}
        \begin{pmatrix}
            1 & 9 & 1 \\
            0 & -44 & 0 \\
            0 & 0 & 0
        \end{pmatrix} \\
        \Rightarrow& X = t\begin{pmatrix}
            -1 \\ 0 \\ 1
        \end{pmatrix}
        \Rightarrow \vec{u}_1 = \begin{pmatrix}
            -1 / \sqrt{2} \\ 0 \\ 1 / \sqrt{2}
        \end{pmatrix}
    \end{align*}
    Now for $\lambda = 4$:
    \begin{align*}
        A - 4I = \begin{pmatrix}
            -3 & 1 & 5 \\
            1 & 1 & 1 \\
            5 & 1 & -3
        \end{pmatrix}
        &\xrightarrow{R_2 \leftrightarrow R_1}
        \begin{pmatrix}
            1 & 1 & 1 \\
            -3 & 1 & 5 \\
            5 & 1 & -3
        \end{pmatrix}
        \xrightarrow[R_3 \to R_3 - 5R_1]{R_2 \to R_2 + 3R_1}
        \begin{pmatrix}
            1 & 1 & 1 \\
            0 & 4 & 8 \\
            0 & -4 & -8
        \end{pmatrix}
        \xrightarrow{R_3 \to R_3 + R_2}
        \begin{pmatrix}
            1 & 1 & 1 \\
            0 & 4 & 8 \\
            0 & 0 & 0 
        \end{pmatrix} \\
        \Rightarrow& X = t\begin{pmatrix}
            1 \\ -2 \\ 1
        \end{pmatrix} 
        \Rightarrow \vec{u}_2 = \begin{pmatrix}
            1 / \sqrt{6} \\
            -2 / \sqrt{6} \\
            1 / \sqrt{6}
        \end{pmatrix}
    \end{align*}
    For $\lambda = 7$:
    \begin{align*}
        A - 7I = &\begin{pmatrix}
            -6 & 1 & 5 \\
            1 & -2 & 1 \\
            5 & 1 & -6 
        \end{pmatrix}
        \xrightarrow{R_2 \leftrightarrow R_1}
        \begin{pmatrix}
            1 & -2 & 1 \\
            -6 & 1 & 5 \\
            5 & 1 & -6 
        \end{pmatrix}
        \xrightarrow[R_3 \to R_3 - 5R_1]{R_2 \to R_2 + 6R_1}
        \begin{pmatrix}
            1 & -1 & 1 \\
            0 & -11 & 11 \\
            0 & 11 & -11 
        \end{pmatrix} \\
        \xrightarrow{R_3 \to R_3 + R_2}
        &\begin{pmatrix}
            1 & -1 & 1 \\
            0 & -11 & 11 \\
            0 & 0 & 0 
        \end{pmatrix}
        \Rightarrow X = t\begin{pmatrix}
            1 \\ 1 \\ 1
        \end{pmatrix}
        \Rightarrow \vec{u}_3 = \begin{pmatrix}
            1 / \sqrt{3} \\ 1 / \sqrt{3} \\ 1 / \sqrt{3}
        \end{pmatrix}
    \end{align*}
    Where $\vec{u}_1, \vec{u}_2, \vec{u}_3$ are unit vectors in the respective eigenspaces. We can now form $P$ and $D$:
    \[
        \boxed{
            P = \begin{pmatrix}
                -1/\sqrt{2} & 1/\sqrt{6} & 1/\sqrt{3} \\
                0 & -2/\sqrt{6} & 1/\sqrt{3} \\
                1/\sqrt{2} & 1/\sqrt{6} & 1/\sqrt{3}
            \end{pmatrix}, 
            \quad D = \begin{pmatrix}
                -4 & 0 & 0 \\
                0 & 4 & 0 \\
                0 & 0 & 7
            \end{pmatrix}
        } 
    \]
\end{solution}

\newpage 

\colorbox{red}{\underline{\#5 [7.2]:}} Find the matrix of the quadratic form. Assume $\vec{x} \in \RR^{3}$.
\begin{parts}
    \part $3x_1^2 + 2x_2^2-5x_3^2-6x_{1}x_{2} + 8x_{1}x_{3} - 4x_{2}x_{3}$
    \begin{solution}
        Denote the above quadratic form as $Q\left(\vec{x}\right)$. The coefficients of the quadratic terms correspond to the main diagonal of the matrix. The coefficients of the cross-product terms correspond to twice the $(i, j)$ and $(j, i)$ entries. With this in mind we can form the matrix as follows, 
        \[
            \boxed{
                \begin{pmatrix}
                    3 & -3 & 4 \\
                    -3 & 2 & -2 \\
                    4 & -2 & -5
                \end{pmatrix}
            } 
        \]
    \end{solution}
    \part $6x_{1}x_{2} + 4x_{1}x_{3} - 10x_{2}x_{3}$
    \begin{solution}
        We can use similar logic to part (a) while noticing that there are no quadratic terms so all the diagonal entries are $0$. Thus, the corresponding matrix is
        \[
            \boxed{
                \begin{pmatrix}
                    0 & 3 & 2 \\
                    3 & 0 & -5 \\
                    2 & -5  & 0
                \end{pmatrix}
            } 
        \]
    \end{solution}
\end{parts}

\colorbox{red}{\underline{\#11 [7.2]:}} Classify the quadratic form $Q\left(\vec{x}\right) = 2x_{1}^{2} - 4x_{1}x_{2} - x_{2}^{2}$. Then make a change of variable, $\vec{x} = P\vec{y}$, that transforms the quadratic form into one with no cross-product term and write the new quadratic form.
\begin{solution}
    Using similar logic as in problem (5) we can form the corresponding matrix as 
    \[
        A = \begin{pmatrix}
            2 & -2 \\
            -2 & -1
        \end{pmatrix}  
    \]
    We can now find the characteristic equation of $A$ (using the known characteristic equation for a $2\times\,2$ matrix) and use it find the eigenvalues:
    \[
        \lambda^2 - \lambda -6 \Rightarrow \left(\lambda - 3\right)\left(\lambda + 2\right) = 0 \Rightarrow \lambda = 3, -2
    \]
    Since the eigenvalues of $A$ are both positive and negative, we can classify $Q$ as \textbf{indefinite}. To find the change of variable we must first find the eigenvectors of $A$: 

    \begin{minipage}{0.4\linewidth}
        \underline{For $\lambda = 3$:}
        \begin{align*}
            A - 3I = &\begin{pmatrix}
                -1 & -2 \\
                -2 & -4
            \end{pmatrix} 
            \xrightarrow{R_2 \to R_2 - 2R_1}
            \begin{pmatrix}
                -1 & -2 \\
                0 & 0 
            \end{pmatrix} \\
            \Rightarrow& X = t\begin{pmatrix}
                -2 \\ 1
            \end{pmatrix}
            \Rightarrow \vec{u}_1 = \begin{pmatrix}
                -2/\sqrt{3} \\ 1/\sqrt{3}
            \end{pmatrix}
        \end{align*}
    \end{minipage}
    \hfill 
    \begin{minipage}{0.4\linewidth}
        \underline{For $\lambda = -2$:}
        \begin{align*}
            A + 2I &\begin{pmatrix}
                4 & -2 \\
                -2 & 1
            \end{pmatrix}
            \xrightarrow{R_2 \to 2R_2 + R_1}
            \begin{pmatrix}
                4 & -2 \\ 
                0 & 0 
            \end{pmatrix} \\
            \Rightarrow& X = t\begin{pmatrix}
                1 \\ 2
            \end{pmatrix}
            \Rightarrow \vec{u}_2 = \begin{pmatrix}
                1/\sqrt{3} \\ 2/\sqrt{3}
            \end{pmatrix}
        \end{align*}
    \end{minipage}
    Using the unit vectors $\vec{u}_1$ and $\vec{u}_2$, as well as $\lambda_1 = 3$ and $\lambda_2 = -2$ we can form $P$ (since the set $\braces{\vec{u}_1, \vec{u}_2}$ is orthogonal) and $D$: 
    \[
        P = \begin{pmatrix}
            -2/\sqrt{3} & 1/\sqrt{3} \\
            1/\sqrt{3} & 2/\sqrt{3}
        \end{pmatrix}, 
        \quad D = \begin{pmatrix}
            3 & 0 \\
            0 & -2
        \end{pmatrix}.
    \]
    Performing the change of variables $\vec{x} = P\vec{y}$ gives the new quadratic form of 
    \[
        \vec{y}^{T}D\vec{y} = \boxed{3y_1^2-2y_2^2}
    \]
\end{solution}
\end{document}