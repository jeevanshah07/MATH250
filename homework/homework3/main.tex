\documentclass{exam}
\usepackage{amsmath}
\usepackage{nicematrix}
\usepackage{extpfeil}
\usepackage{amsthm}
\usepackage{gensymb}
\usepackage{amsfonts}
\usepackage{tikz}
\usepackage{pgfplots}
\usepackage{array}
\usepackage{hyperref}
\usepackage{nameref}
\usepackage[x11names]{xcolor}
\usepackage[most]{tcolorbox}

\pgfplotsset{compat=1.18}
\printanswers

\newcommand{\ZZ}{\mathbb Z}
\newcommand{\QQ}{\mathbb Q}
\newcommand{\NN}{\mathbb N}
\newcommand{\RR}{\mathbb R}
\newcommand{\braces}[1]{\ensuremath{\left\{#1 \right\}}}
\newcommand{\Span}[1]{\ensuremath\text{Span}\braces{#1}}
\newcommand{\ee}{\mathbf{e}}

\hypersetup{colorlinks=true, linktoc=section, linkcolor=blue}

\pagestyle{headandfoot}
\firstpageheadrule
\runningheadrule
\firstpageheader{Prof. Shtelen \\ Linear Algebra}{Homework 3}{Jeevan Shah}
\runningheader{Linear Algebra \\ Homework 2}{}{Shah}
\firstpagefooter{}{}{}
\runningfooter{ }{\thepage}{ }

\printanswers

\newenvironment{amatrix}[1]{%
  \left(\begin{array}{@{}*{#1}{c}|c@{}}
}{%
  \end{array}\right)
}
% 23 41
% 3 13
\begin{document}
\underline{\#23 [2.3]:} Can a square matrix with identical columns be invertible? Why or why not?
\begin{solution}
    Any square matrix with at least two identical columns \textbf{cannot} be invertible. This is because if two columns are identical then there exist non-trivial solutions to the system $A\vec{x}=\vec{0}$: Let 
    \[
        A = \begin{pmatrix}
            \left(\vec{a_1}\right) & \left(a_2\right) & \cdots & \left(a_n\right)
        \end{pmatrix} 
    \]
    be an $n \times n$ matrix. Now assume that two columns are identical or, $\vec{a_i} = \vec{a_j}$ for $i \neq j$. Then for the linear combination 
    \[
        c_1\vec{a_1} + c_2\vec{a_2} + \cdots + c_n\vec{a_n} = \vec{0}
    \]
    we have the solution $c_i = 1, c_j=-1$ and $c_k=0\,(\forall k \neq i, j)$. Thus, there exists a non-trivial solution, and so the columns of $A$ are linearly dependent and thus $A$ is \textbf{not} invertible.
\end{solution}

\underline{\#41 [2.3]:} Let $\mathcal{T}:\RR^{2} \to \RR^{2}$ be a linear transformation as defined below. Show that $\mathcal{T}$ is invertible and find a formula for $\mathcal{T}^{-1}$. 
\[
    \mathcal{T}\left(x_1, x_2\right) = \left(-5x_1 + 9x_2, 4x_1 - 7x_2\right)
\]
\begin{solution}
    We can begin by finding the matrix associated with $\mathcal{T}$, 
    \[
        A = \begin{pmatrix}
            \left(\mathcal{T}\left(E_1\right)\right) & \left(\mathcal{T}\left(E_2\right)\right)
        \end{pmatrix} 
        = 
        \begin{pmatrix}
            -5 & 9 \\ 4 & -7
        \end{pmatrix}
    \]
    Then, $\det\left(A\right) = -1 \neq 0$ so $A$ is invertible and thus $\mathcal{T}$ is invertible. Let $\mathcal{S}\left(\vec{x}\right) = \mathcal{T}^{-1}\left(\vec{x}\right)$. Then, 
    \[
        \mathcal{S}\left(\vec{x}\right) = A^{-1}\vec{x} = 
        \begin{pmatrix}
            7 & 9 \\ 4 & 5
        \end{pmatrix}\vec{x}
    \]
    Note that we were able to quickly find $A^{-1}$ using the formula for $2 \times 2$ matricies: 
    \[
        A = \begin{pmatrix}
            a & b \\ c & d 
        \end{pmatrix}
        \Rightarrow
        A^{-1} = \frac{1}{\det\left(A\right)}\begin{pmatrix}
            d & -b \\ -c & a
        \end{pmatrix}
    \]
\end{solution}


\underline{\#3 [3.1]:} Compute the determinant for the given matrix below using a cofactor expansion across the first row. Then, compute the determinant by a cofactor expansion down the second column.
\[
    \begin{vmatrix}
        2 & -2 & 3 \\
        3 & 1 & 2 \\
        1 & 3 & -1
    \end{vmatrix} 
\]
\begin{solution}
    Along the first row we have:
    \[
        \Delta = 2\begin{vmatrix}
            1 & 2 \\ 3 & -1 
        \end{vmatrix}
        - (-2)\begin{vmatrix}
            3 & 2 \\ 1 & -1
        \end{vmatrix}
        + 3\begin{vmatrix}
            3 & 1 \\ 1 & 3
        \end{vmatrix}
        = 2(-1 - 6) + 2(-3 - 2) + 3(9-1) = \boxed{0}
    \]
    Along the second column we have: 
    \[
        \Delta = (-1)^{1+2}(-2)\begin{vmatrix}
            3 & 2 \\ 1 & -1 
        \end{vmatrix}
        + (-1)^{2+2}(1)\begin{vmatrix}
            2 & 3 \\ 1 & -1 
        \end{vmatrix}
        + (-1)^{3+2}(3)\begin{vmatrix}
            2 & 3 \\ 3 & 2
        \end{vmatrix}
        = 2(-3 - 2) + 1(-2 - 3) - 3(4 - 9) = \boxed{0}
    \]
\end{solution}

\underline{\#13 [3.1]:} Compute the determinant for the given matrix below by cofactor expansions. At each step, choose a row or column that involves the least amount of computation. 
\[
    \begin{vmatrix}
        4 & 0 & -7 & 3 & -5 \\
        0 & 0 & 2 & 0 & 0 \\
        7 & 3 & -6 & 4 & -8 \\
        5 & 0 & 5 & 2 & -3 \\
        0 & 0 & 9 & -1 & 2
    \end{vmatrix}
\]
\begin{solution}
    \[
        \Delta = (-1)^{3+2}(3)\begin{vmatrix}
            4 & -7 & 3 & -5 \\
            0 & 2 & 0 & 0 \\
            5 & 5 & 2 & -3 \\
            0 & 9 & -1 & 2
        \end{vmatrix} 
        = -3(-1)^{2+2}(2)\begin{vmatrix}
            4 & 3 & -5 \\
            5 & 2 & -3 \\
            0 & -1 & 2
        \end{vmatrix}
        = -6\left(4(1) - 3(10) + (-5)(-5)\right) = \boxed{6}
    \]
\end{solution}
\end{document}